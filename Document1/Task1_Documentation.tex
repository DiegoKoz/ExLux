\documentclass[]{article}
\usepackage{lmodern}
\usepackage{amssymb,amsmath}
\usepackage{ifxetex,ifluatex}
\usepackage{fixltx2e} % provides \textsubscript
\ifnum 0\ifxetex 1\fi\ifluatex 1\fi=0 % if pdftex
  \usepackage[T1]{fontenc}
  \usepackage[utf8]{inputenc}
\else % if luatex or xelatex
  \ifxetex
    \usepackage{mathspec}
  \else
    \usepackage{fontspec}
  \fi
  \defaultfontfeatures{Ligatures=TeX,Scale=MatchLowercase}
\fi
% use upquote if available, for straight quotes in verbatim environments
\IfFileExists{upquote.sty}{\usepackage{upquote}}{}
% use microtype if available
\IfFileExists{microtype.sty}{%
\usepackage{microtype}
\UseMicrotypeSet[protrusion]{basicmath} % disable protrusion for tt fonts
}{}
\usepackage[margin=1in]{geometry}
\usepackage{hyperref}
\hypersetup{unicode=true,
            pdftitle={Task1: Documentation},
            pdfauthor={Diego Kolzowski},
            pdfborder={0 0 0},
            breaklinks=true}
\urlstyle{same}  % don't use monospace font for urls
\usepackage{color}
\usepackage{fancyvrb}
\newcommand{\VerbBar}{|}
\newcommand{\VERB}{\Verb[commandchars=\\\{\}]}
\DefineVerbatimEnvironment{Highlighting}{Verbatim}{commandchars=\\\{\}}
% Add ',fontsize=\small' for more characters per line
\usepackage{framed}
\definecolor{shadecolor}{RGB}{248,248,248}
\newenvironment{Shaded}{\begin{snugshade}}{\end{snugshade}}
\newcommand{\AlertTok}[1]{\textcolor[rgb]{0.94,0.16,0.16}{#1}}
\newcommand{\AnnotationTok}[1]{\textcolor[rgb]{0.56,0.35,0.01}{\textbf{\textit{#1}}}}
\newcommand{\AttributeTok}[1]{\textcolor[rgb]{0.77,0.63,0.00}{#1}}
\newcommand{\BaseNTok}[1]{\textcolor[rgb]{0.00,0.00,0.81}{#1}}
\newcommand{\BuiltInTok}[1]{#1}
\newcommand{\CharTok}[1]{\textcolor[rgb]{0.31,0.60,0.02}{#1}}
\newcommand{\CommentTok}[1]{\textcolor[rgb]{0.56,0.35,0.01}{\textit{#1}}}
\newcommand{\CommentVarTok}[1]{\textcolor[rgb]{0.56,0.35,0.01}{\textbf{\textit{#1}}}}
\newcommand{\ConstantTok}[1]{\textcolor[rgb]{0.00,0.00,0.00}{#1}}
\newcommand{\ControlFlowTok}[1]{\textcolor[rgb]{0.13,0.29,0.53}{\textbf{#1}}}
\newcommand{\DataTypeTok}[1]{\textcolor[rgb]{0.13,0.29,0.53}{#1}}
\newcommand{\DecValTok}[1]{\textcolor[rgb]{0.00,0.00,0.81}{#1}}
\newcommand{\DocumentationTok}[1]{\textcolor[rgb]{0.56,0.35,0.01}{\textbf{\textit{#1}}}}
\newcommand{\ErrorTok}[1]{\textcolor[rgb]{0.64,0.00,0.00}{\textbf{#1}}}
\newcommand{\ExtensionTok}[1]{#1}
\newcommand{\FloatTok}[1]{\textcolor[rgb]{0.00,0.00,0.81}{#1}}
\newcommand{\FunctionTok}[1]{\textcolor[rgb]{0.00,0.00,0.00}{#1}}
\newcommand{\ImportTok}[1]{#1}
\newcommand{\InformationTok}[1]{\textcolor[rgb]{0.56,0.35,0.01}{\textbf{\textit{#1}}}}
\newcommand{\KeywordTok}[1]{\textcolor[rgb]{0.13,0.29,0.53}{\textbf{#1}}}
\newcommand{\NormalTok}[1]{#1}
\newcommand{\OperatorTok}[1]{\textcolor[rgb]{0.81,0.36,0.00}{\textbf{#1}}}
\newcommand{\OtherTok}[1]{\textcolor[rgb]{0.56,0.35,0.01}{#1}}
\newcommand{\PreprocessorTok}[1]{\textcolor[rgb]{0.56,0.35,0.01}{\textit{#1}}}
\newcommand{\RegionMarkerTok}[1]{#1}
\newcommand{\SpecialCharTok}[1]{\textcolor[rgb]{0.00,0.00,0.00}{#1}}
\newcommand{\SpecialStringTok}[1]{\textcolor[rgb]{0.31,0.60,0.02}{#1}}
\newcommand{\StringTok}[1]{\textcolor[rgb]{0.31,0.60,0.02}{#1}}
\newcommand{\VariableTok}[1]{\textcolor[rgb]{0.00,0.00,0.00}{#1}}
\newcommand{\VerbatimStringTok}[1]{\textcolor[rgb]{0.31,0.60,0.02}{#1}}
\newcommand{\WarningTok}[1]{\textcolor[rgb]{0.56,0.35,0.01}{\textbf{\textit{#1}}}}
\usepackage{graphicx,grffile}
\makeatletter
\def\maxwidth{\ifdim\Gin@nat@width>\linewidth\linewidth\else\Gin@nat@width\fi}
\def\maxheight{\ifdim\Gin@nat@height>\textheight\textheight\else\Gin@nat@height\fi}
\makeatother
% Scale images if necessary, so that they will not overflow the page
% margins by default, and it is still possible to overwrite the defaults
% using explicit options in \includegraphics[width, height, ...]{}
\setkeys{Gin}{width=\maxwidth,height=\maxheight,keepaspectratio}
\IfFileExists{parskip.sty}{%
\usepackage{parskip}
}{% else
\setlength{\parindent}{0pt}
\setlength{\parskip}{6pt plus 2pt minus 1pt}
}
\setlength{\emergencystretch}{3em}  % prevent overfull lines
\providecommand{\tightlist}{%
  \setlength{\itemsep}{0pt}\setlength{\parskip}{0pt}}
\setcounter{secnumdepth}{0}
% Redefines (sub)paragraphs to behave more like sections
\ifx\paragraph\undefined\else
\let\oldparagraph\paragraph
\renewcommand{\paragraph}[1]{\oldparagraph{#1}\mbox{}}
\fi
\ifx\subparagraph\undefined\else
\let\oldsubparagraph\subparagraph
\renewcommand{\subparagraph}[1]{\oldsubparagraph{#1}\mbox{}}
\fi

%%% Use protect on footnotes to avoid problems with footnotes in titles
\let\rmarkdownfootnote\footnote%
\def\footnote{\protect\rmarkdownfootnote}

%%% Change title format to be more compact
\usepackage{titling}

% Create subtitle command for use in maketitle
\newcommand{\subtitle}[1]{
  \posttitle{
    \begin{center}\large#1\end{center}
    }
}

\setlength{\droptitle}{-2em}

  \title{Task 1: Documentation}
    \pretitle{\vspace{\droptitle}\centering\huge}
  \posttitle{\par}
    \author{Diego Kolzowski}
    \preauthor{\centering\large\emph}
  \postauthor{\par}
    \date{}
    \predate{}\postdate{}
  

\begin{document}
\maketitle

The following document will present a brief summary of some key-points from the first-task's workflow.  



\section{First objective}

\textbf{absolute number of articles for the given year}

\subsubsection{Counting criteria}

\begin{itemize}
	\item \emph{first author counting} considers only the first author of each
	publication. This means that each publication is considered only once.
	\item \emph{whole counting} give one credit to every author of each
	publication. This means that each publication is consider as many times
	as authors has. 
	\item \emph{whole-normalized counting} consider all authors
	but distributes one credit between them equally distributed. This means
	that all the authors are consider but each publication is considered as
	one credit.
	\item \emph{complete-normalized counting} consider all
	authors and distributes one credit per publication, but in an unequally
	distributed way.
\end{itemize}



Given that the goal of the task is to count how many articles where
published for Germany in 2010 for STEM, I consider that the \emph{first
author counting} is the best criteria. The final count of the
\emph{whole-normalized counting} and the \emph{complete-normalized
counting} would give the same result. \emph{whole counting} is not
useful as it would inflate the result we are interested in.

To achieve this result I remove the duplicated rows by \texttt{ut}.

\textbf{Result}
\begin{verbatim}
>> 74286
\end{verbatim}

\section{Second objective}


\textbf{clean and re-code the variable ``organization''}



First, to identify all articles which have at least one author who is
affiliated to a university I need to remove those rows where no
organization is defined.

Then, I remove all numbers, punctuation marks, accents, etc. For this, I define the following cleaning function:

\begin{Shaded}
\begin{Highlighting}[]
\NormalTok{text_cleaner <-}\StringTok{ }\ControlFlowTok{function}\NormalTok{(x)\{}
  \CommentTok{#replace numbers}
\NormalTok{  x <-}\StringTok{  }\NormalTok{stringr}\OperatorTok{::}\KeywordTok{str_replace_all}\NormalTok{(x, stringr}\OperatorTok{::}\KeywordTok{regex}\NormalTok{(}\StringTok{"[0-9]*"}\NormalTok{),}\StringTok{""}\NormalTok{)}
  \CommentTok{#replace replace punctuation}
\NormalTok{  x <-}\StringTok{ }\NormalTok{stringr}\OperatorTok{::}\KeywordTok{str_replace_all}\NormalTok{(x,stringr}\OperatorTok{::}\KeywordTok{regex}\NormalTok{(}\StringTok{"(}\CharTok{\textbackslash{}\textbackslash{}}\StringTok{+|}\CharTok{\textbackslash{}\textbackslash{}}\StringTok{-|}\CharTok{\textbackslash{}\textbackslash{}}\StringTok{=|}\CharTok{\textbackslash{}\textbackslash{}}\StringTok{:|;|}\CharTok{\textbackslash{}\textbackslash{}}\StringTok{.|,|_|}\CharTok{\textbackslash{}\textbackslash{}}\StringTok{?|¿|}\CharTok{\textbackslash{}\textbackslash{}}\StringTok{!|¡|}\CharTok{\textbackslash{}\textbackslash{}\textbackslash{}\textbackslash{}}\StringTok{|}\CharTok{\textbackslash{}\textbackslash{}}\StringTok{(|}\CharTok{\textbackslash{}\textbackslash{}}\StringTok{)|}\CharTok{\textbackslash{}\textbackslash{}}\StringTok{||}\CharTok{\textbackslash{}\textbackslash{}}\StringTok{^|}\CharTok{\textbackslash{}\textbackslash{}}\StringTok{>|}\CharTok{\textbackslash{}\textbackslash{}}\StringTok{<|}\CharTok{\textbackslash{}\textbackslash{}}\StringTok{/|#|}\CharTok{\textbackslash{}\textbackslash{}}\StringTok{$|%|&|}\CharTok{\textbackslash{}"}\StringTok{|}\CharTok{\textbackslash{}\textbackslash{}}\StringTok{*|}\CharTok{\textbackslash{}\textbackslash{}}\StringTok{\{|}\CharTok{\textbackslash{}\textbackslash{}}\StringTok{\}|`|}\CharTok{\textbackslash{}\textbackslash{}}\StringTok{[|´|}\CharTok{\textbackslash{}\textbackslash{}}\StringTok{]|@|¨|°|ª)"}\NormalTok{),}\StringTok{""}\NormalTok{)}
  \CommentTok{#replace repeted line breaks and carriage returns}
\NormalTok{  x <-}\StringTok{ }\NormalTok{stringr}\OperatorTok{::}\KeywordTok{str_replace_all}\NormalTok{(x,}\StringTok{'(}\CharTok{\textbackslash{}r\textbackslash{}n}\StringTok{)|(}\CharTok{\textbackslash{}n\textbackslash{}r}\StringTok{)'}\NormalTok{,}\StringTok{'}\CharTok{\textbackslash{}n}\StringTok{'}\NormalTok{) }\OperatorTok
\StringTok{    }\NormalTok{stringr}\OperatorTok{::}\KeywordTok{str_replace_all}\NormalTok{(}\StringTok{'}\CharTok{\textbackslash{}n}\StringTok{+'}\NormalTok{,}\StringTok{'}\CharTok{\textbackslash{}n}\StringTok{'}\NormalTok{) }\OperatorTok
\StringTok{    }\NormalTok{stringr}\OperatorTok{::}\KeywordTok{str_replace_all}\NormalTok{(}\StringTok{'}\CharTok{\textbackslash{}r}\StringTok{+'}\NormalTok{,}\StringTok{'}\CharTok{\textbackslash{}r}\StringTok{'}\NormalTok{) }\OperatorTok
\StringTok{    }\NormalTok{stringr}\OperatorTok{::}\KeywordTok{str_replace_all}\NormalTok{(}\StringTok{'(}\CharTok{\textbackslash{}r\textbackslash{}n}\StringTok{)|(}\CharTok{\textbackslash{}n\textbackslash{}r}\StringTok{)'}\NormalTok{,}\StringTok{'}\CharTok{\textbackslash{}n}\StringTok{'}\NormalTok{) }\OperatorTok
\StringTok{    }\NormalTok{stringr}\OperatorTok{::}\KeywordTok{str_replace_all}\NormalTok{(}\StringTok{'}\CharTok{\textbackslash{}n}\StringTok{+'}\NormalTok{,}\StringTok{'}\CharTok{\textbackslash{}n}\StringTok{'}\NormalTok{)}
  \CommentTok{#to lowercase}
\NormalTok{  x <-}\StringTok{ }\KeywordTok{str_to_lower}\NormalTok{(x)}
  \CommentTok{#remove accents}
\NormalTok{  x <-}\StringTok{ }\NormalTok{stringi}\OperatorTok{::}\KeywordTok{stri_trans_general}\NormalTok{(x,}\StringTok{"Latin-ASCII"}\NormalTok{)}
  \KeywordTok{return}\NormalTok{(x)}
\NormalTok{\}}
\end{Highlighting}
\end{Shaded}

After this, I need to find if their organization is a university or a
private institution. A brief inspection of the dataset shows that the
keyword for universities is \emph{`univ'}. I filter the results that
contain `univ' as part of the organization name (after cleaning)


In order to address the problem of multiple names, I summarize the
information by suborganization and city:


\begin{verbatim}

##    suborganizations   city     len_org orgs                                
##    <chr>              <chr>      <int> <chr>                               
##  1 Inst Med Virol     frankfu~      12 clin goethe univ  | goethe univ fra~
##  2 Dept Surg          munich         9 klinikum univ munchen  | lm univ mu~
##  3 Dept Urol          munich         8 klinikum univ muenchen  | klinikum ~
##  4 Dept Internal Med~ regensb~       7 hosp univ regensburg  | regensburg ~
##  5 Inst Klin Radiol   munich         7 klinikum ludwig maximilians univ mu~
##  6 Inst Pathol        freiburg       7 univ freiburg  | univ freiburg klin~
##  7 Inst Pathol        mainz          7 johannes gutenberg univ hosp  | joh~
##  8 Dept Cardiovasc S~ freiburg       6 med univ clin  | univ clin freiburg~
##  9 Dept Chem          munich         6 lmu univ munich  | ludwig maximilia~
## 10 Dept Diagnost Rad~ freiburg       6 med phys univ hosp  | univ freiburg~
## # ... with 23,415 more rows
\end{verbatim}


The majority of the dataset (91\% of the suborganization \& city
pairs) have one Organization per suborganization \& city.

For the rest of the dataset where the suborganization is defined, the
organization's names only repeat few times and by a brief inspection,
they all refer to the same institution. This means, the couple
suborganization \& city can be used for unifying the organization name
field \footnote{note: for a real workflow situation a much more careful
  analysis should be made in order to avoid unifying different
  organizations}


When we analyse the data with no suborganizations, we found much more
repetition by city and it is not clear that they all belong to the same
organization:


\begin{verbatim}
##    suborganizations city     len_org orgs                                  
##    <chr>            <chr>      <int> <chr>                                 
##  1 <NA>             munich        41 bw univ munich  | chirurg univ klin  ~
##  2 <NA>             hamburg       35 dvgw forsch stelle tech univ hamburg ~
##  3 <NA>             heidelb~      35 chirurg univ klin  | heidelberg univ ~
##  4 <NA>             berlin        25 alice salomon univ appl sci  | benjam~
##  5 <NA>             freiburg      25 anasthesiol univ klin freiburg  | chi~
##  6 <NA>             tubingen      23 childrens univ hosp  | orthopad univ ~
##  7 <NA>             frankfu~      20 clin johann wolfgang goethe univ  | g~
##  8 <NA>             leipzig       20 hno univ klin leipzig  | inst univ le~
##  9 <NA>             ulm           20 orthopad univ klinikum rku  | univ ap~
## 10 <NA>             essen         19 duisburg essen univ  | folkwang univ ~
## # ... with 144 more rows
\end{verbatim}


\subsubsection{Unification of the organization-label}

The proposed workflow is the following:

\begin{enumerate}
\def\labelenumi{\arabic{enumi}.}
\tightlist
\item
  For the data with suborganization \& city: count the number of times
  each organization name is used. When there is a tie, I will choose the
  shortest one.
\item
  Define a codebook which associates the most used name with the others.
\item
  recode the names based on the codebook.
\end{enumerate}

The \textit{codebook} looks like the following table:

\begin{verbatim}
## # A tibble: 1,328 x 2
##    organization                         organization_new                   
##    <chr>                                <chr>                              
##  1 free univ berlin                     free univ berlin                   
##  2 med univ klin freiburg               med univ klin freiburg             
##  3 johannes gutenberg univ mainz        johannes gutenberg univ mainz      
##  4 chirurg univ klin freiburg           univ freiburg                      
##  5 univ freiburg                        univ freiburg                      
##  6 univ klinikum freiburg i br          univ freiburg                      
##  7 univ klinikum heidelberg             univ klinikum heidelberg           
##  8 univ marburg                         univ marburg                       
##  9 univ klin kinder & jugendmed tubing~ univ klin kinder  jugendmed tubing~
## 10 univ klin kinder & jugendmed         univ klin kinder  jugendmed        
## # ... with 1,318 more rows
\end{verbatim}

I need to add those organizations that don't appear in the codebook because they don't have suborganization. For this, I join the codebook with the original dataset, and filter those cases where \textit{organization\_new} is empty. I append those cases and assign the cleaned organization name as the value for \textit{organization\_new}

\medskip

The final cleaning is to unify the use of words:

\begin{itemize}
\tightlist
\item
  remove \textit{univ} and derivatives from the text and re-add it at the end
  (normalized way) in order to recongnize they are from universities
\item
  normalize the derivatives of \textit{klinikum}
\item
  final adjustments of other variations
\end{itemize}



\begin{verbatim}
## Original organization names: 1721

## before extra cleaning:1339 new names

## after extra cleaning: 1284 new names

\end{verbatim}

The number of organizations names is reduced in 437


The final step is to recode the original data and save the results.

\hypertarget{final-notes}{%
\section{Final notes}\label{final-notes}}

\hypertarget{program-choice}{%
\subsubsection{Program choice}\label{program-choice}}

I decided to use R as it allows to use powerfull libraries for Text
Mining, and hence for the normalization process, and also embed the
documentation in the code.


\end{document}
